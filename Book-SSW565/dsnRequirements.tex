\chapter{Requirements \\
\small{\textit{-- Author Name}}
\index{introduction} 
\index{Chapter!Requirements}
\label{Chapter::Requirements}}

\section{Stakeholders}

People or roles who are affected, in some way, by a system and so who can contribute requirements or
knowledge to help you understand the requirements:

\subsection{Customers} 
Clients and users. Who are they, why do they use your system

\subsection{Sponsors}

\subsection{Engineering and Technical Persons}

\subsection{Regulators}

\subsection{Third Parties}

\subsection{Competitors} 

Systems which provide similar functions.

\section{Key Concepts}

List the key concepts and their definitions that relate to your project. These concepts and terms
should be used consistently throughout this document and in your project.  This can point out to the Glossary chapter.

\section{User Requirements}
Abstract statements written in natural language with accompanying informal diagrams. You may
use user stories or simple use case diagram(s).
The user requirements should be numbered and linked to use-cases and user-stories.  Here is an 
example.


\small
\begin{longtable}{|p{11cm}|p{1.5cm}|p{2cm}|}
%\center
\caption{User Requirements Table \label{Table::RequirementsUser}}\\
\hline
\textbf{Requirement} & \textbf{Priority} & \textbf{Use Case(s)} \\
\hline 
\endhead

\begin{reqkQuality}[
\RequirementName{reqkQuality}{reqqFirstQualityRequirement}]
\RequirementLabel{reqkQuality}{reqqFirstQualityRequirement}
The system shall perform the task in less than one second.
\end{reqkQuality}
& 
\gls{must}
&
\UseCaseReference{ucFirstUseCase}
\\ 
\hline

\begin{reqkFunctional}[
\RequirementName{reqkFunctional}{reqfFirstFunctionalRequirement}]
\RequirementLabel{reqkFunctional}{reqfFirstFunctionalRequirement}
The system shall perform the task of adding 1+1.
\end{reqkFunctional}
& 
\gls{should}
&
\UseCaseReference{ucFirstUseCase}
\\ 
\hline

\begin{reqkConstraint}[
\RequirementName{reqkConstraint}{reqcFirstConstraintRequirement}]
\RequirementLabel{reqkConstraint}{reqcFirstConstraintRequirement}
The system shall not interfere with local Wi-Fi communication channels.
\end{reqkConstraint}
& 
\gls{could}
&
\UseCaseReference{ucFirstUseCase}
\\ 
\hline

\begin{reqkInterface}[
\RequirementName{reqkInterface}{reqiFirstInterfaceRequirement}]
\RequirementLabel{reqkInterface}{reqiFirstInterfaceRequirement}
The system shall have a C++ API.
\end{reqkInterface}
& 
\gls{would}
&
\UseCaseReference{ucFirstUseCase}
\\ 

\hline
\begin{reqkBusiness}[
\RequirementName{reqkBusiness}{reqbFirstBusinessRequirement}]
\RequirementLabel{reqkBusiness}{reqbFirstBusinessRequirement}
The system shall be FDA approved.
\end{reqkBusiness}
& 
\gls{would}
&
\UseCaseReference{ucFirstUseCase}
\\ 
\hline

\end{longtable}

At some point in your document, you will need to define your use case.
For this to all work, you also need to define {\tt newtheorem} for the requirements and 
use cases, in your root manual.tex.

\section{System (Constraints) Requirements}
More detailed descriptions of the services and constraints from the perspective of the system to
meet the user requirements. Should be structured and precise. More detailed that describe the
user requirements and focus on the basic flow, alternative flow, exceptions, pre-conditions, and
post-conditions, and special requirements for each abstract user requirements. May use use-
case template for this.  System requirements should be numbered and traced back to the user requirements.

\section{Non-functional (Quality) Requirements}
Any important non-functional requirements for your project.  These would include any 
performance and quality requirements (i.e. numbers, such as bandwidth, time, speed, rates, etc.).
Can add here any type of a multitude of quality requirements that define, in large part, the 
system architecture.

\section{Domain (Business) Requirements}
Business rules and regulations that impact what the system does.

\section{Other Requirement Types}
If there is a reason to define a new, non-overlapping requirement class, you can create
more requirement classes as needed, but be careful to not create too many.

