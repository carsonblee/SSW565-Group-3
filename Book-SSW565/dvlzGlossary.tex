\label{Chapter::Glossary}
\newglossaryentry{must}
{	name={MustHave},
	description={This defines the first highest priority requirement.
	All of the tasks, requirements, or anything that is marked this way are
	build in the current version}
}

\newglossaryentry{should}
{	name={ShouldHave},
	description={This defines the second highest priority requirement. The system should implement 
	all of the tasks, requirements, or anything that is marked this way, but if 
	resources are limited, it can be left out of the current version.
	Build in next version}
}

\newglossaryentry{could}
{	name={CouldHave},
	description={This defines the third highest priority requirement.The system could implement 
	all of the tasks, requirements, or anything that is marked this way, but if 
	resources are limited, it can be left out of the current and next version.
	Build in two versions from now}
}

\newglossaryentry{would}
{	name={WouldHave},
	description={This defines the lowest priority requirement.  The system would like to implement 
all of the tasks, requirements, or anything that is marked this way, but only
if resources are available. It can be left out of all future versions}
}

\newglossaryentry{course}
{	name={course},
	description={A specific instance (with a time, location, and professor) of a class offered at a university}
}

\newglossaryentry{stakeholder}
{   name={stakeholder},
        description={A particular member/party that has a particular interest and/or investment in the outcome of an organization, business, or project}
}

\newglossaryentry{source}
{	name={source},
	description={The location or actor that sends the stimulus. \\ \textit{Example: "Unexpected server failure" → Addressed by Active Redundancy}}
}

\newglossaryentry{stimulus}
{	name={stimulus},
	description={The acting factor or input on the system}
}

\newglossaryentry{environment}
{	name={environment},
	description={The operating condition of the system when the stimulus occurs}
}

\newglossaryentry{artifact}
{	name={artifact},
	description={The system components used to implement availability tactics. \\ \textit{Example: "Monitoring tools" → Supports Heartbeat, and Ping/Echo}}
}

\newglossaryentry{response}
{	name={availability response},
	description={Specific tactics used to detect, prevent, or recover from faults. \\ \textit{Example: Detect faults: Monitor, Heartbeat, Exception Detection Recover from faults: Rollback, Retry, Reconfiguration Prevent faults: Predictive Model, Removal from Service}}
}

\newglossaryentry{responseMeasure}
{	name={response measure},
	description={The result of applying tactics. \\ \textit{Example: Ensuring system remains available by automatically switching to backup servers (Passive/Active Redundancy), Divide requests on multiple servers, Applying scaling principles}}
}

\newglossaryentry{CRC card}
{   name={CRC card},
        description={A class-responsibility-collaboration card is a brainstorming tool used in the design of object-oriented software}
}