%----------------------------------------------------------------------------------------
%-------------------------------------- HOMEWORK 6a --------------------------------------
%----------------------------------------------------------------------------------------
\section{Assignment 6a \textit{-- Architecture Patterns}}\label{Assignments::6a}
\textbf{Due:} March 10th, 2025 \\
\begin{enumerate}
    \item Create a new chapter in your Overleaf/LaTeX manual and call it: Architecture Patterns.
    \item Add a \verb|\cite{}| to the Bibliography and reference the video link from above.
    \item Watch the video on the Zappos Systems Architecture found in this week's Module on Canvas.  Like Zappos, suppose that you have a very large dataset of shoes with various brands, styles, colors and sizes. In the chapter answer and expand on the following questions:
    \begin{enumerate}
        \item Which pattern should be applied for a a basic search engine capability for your dataset of shoes? Explain why this pattern should be applied.  Include at least one block diagram of your design.
        \item Place your search engine design within a component and connector pattern that could be used to support user access and prepare a brief description of your overall design.
    \end{enumerate}
    \item Using the code from MVC.zip
    \begin{enumerate}
        \item Create the UML diagram of the MVC files
        \item Create an sequence diagram that depicts the sequence of events in the main() program.  Make sure you set break-points and step through the program to clearly map all the function calls.
    \end{enumerate}
    \item \textbf{In class exercise:} Redesign the Oscilloscope using the "MVC" for the user interface and "Pipe and Filter" for the signal processing part.  Assume that the two filters we have are: scpScale (to scale the signal) and scpOffset to offset the signal.  Provide a UML diagram class diagram of the new design.  Use the standard associations: inheritance, composition, aggregation, and label any other associations by name.  Use multiplicities, too.
\end{enumerate}